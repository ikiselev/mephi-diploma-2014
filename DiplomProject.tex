\documentclass[a4paper,12pt]{report}
\usepackage[utf8]{inputenc}
\usepackage{polyglossia}
\setdefaultlanguage{russian}
\usepackage{amssymb,amsfonts,amsmath,mathtext,cite,enumerate,float}



\usepackage[toc]{appendix}
\usepackage{titletoc}
\usepackage{tocloft}
\usepackage{etoolbox}
\usepackage{textcase}
\usepackage{titlesec}


\usepackage{fontspec}
\setmainfont{Times New Roman}



\makeatletter

%chapter - Делаем текст большими буквами и с точками в содержании
\renewcommand*\l@chapter[2]{%
  \ifnum \c@tocdepth >\m@ne
    \addpenalty{-\@highpenalty}%
    \vskip 1.0em \@plus\p@
    \setlength\@tempdima{1.5em}%
    \begingroup
      \parindent \z@ \rightskip \@pnumwidth
      \parfillskip -\@pnumwidth
      \leavevmode
      \advance\leftskip\@tempdima
      \hskip -\leftskip
      \MakeTextUppercase{#1}\nobreak\leaders\hbox{$\m@th
\mkern \@dotsep mu\hbox{.}\mkern \@dotsep
mu$}\hfill
\nobreak\hb@xt@\@pnumwidth{\hss #2}\par
      \penalty\@highpenalty
    \endgroup
  \fi}




%http://tex.stackexchange.com/questions/63390/how-to-decrease-spacing-before-chapter-title
%since \@makechapterhead adds a 50pt space above the title and 40pt after it.
%A different strategy might be to redefine \@makechapterhead yourself:

\def\@makechapterhead#1{%
  %%%%%\vspace*{50\p@}% %%% removed!
  {\parindent \z@ \raggedright
    \normalfont
    \interlinepenalty\@M
    \large \bfseries \MakeUppercase{#1}\par\nobreak
    \vskip 10\p@
  }}

%Со звездочкой
\def\@makeschapterhead#1{%
  %%%%%\vspace*{50\p@}% %%% removed!
  {\parindent \z@ \raggedright
    \normalfont
    \interlinepenalty\@M
    \Large \bfseries \MakeUppercase{#1}\par\nobreak
    \vskip 10\p@
  }}



\titleformat*{\section}{\large\bfseries\itshape}


\makeatother


\usepackage{geometry} % Меняем поля страницы
\geometry{left=1.5cm}% левое поле
\geometry{right=1.5cm}% правое поле
\geometry{top=1cm}% верхнее поле
\geometry{bottom=2cm}% нижнее поле


\renewcommand{\theenumi}{\arabic{enumi}}% Меняем везде перечисления на цифра.цифра
\renewcommand{\labelenumi}{\arabic{enumi}}% Меняем везде перечисления на цифра.цифра
\renewcommand{\theenumii}{.\arabic{enumii}}% Меняем везде перечисления на цифра.цифра
\renewcommand{\labelenumii}{\arabic{enumi}.\arabic{enumii}.}% Меняем везде перечисления на цифра.цифра
\renewcommand{\theenumiii}{.\arabic{enumiii}}% Меняем везде перечисления на цифра.цифра
\renewcommand{\labelenumiii}{\arabic{enumi}.\arabic{enumii}.\arabic{enumiii}.}% Меняем везде перечисления на цифра.цифра


\usepackage{setspace}
\setstretch{1.5}

\setcounter{tocdepth}{1} %n=1 это chapter и section в оглавлении

\parindent=1cm %абзацный отступ


%Здесь переменные. Они вынесутся в отдельный файл
\newcommand{\username}{Иванов Иван Иванович}
\newcommand{\thesisTheme}{Разработка системы управления миром с использованием систем контроля версий}
\newcommand{\projectManager}{Другой Иванов Иван Иванович, аспирант}
\newcommand{\workplace}{НИЯУ МИФИ} %переменные: тема работы, студент и т.д.


\begin{document}

\begin{titlepage}
\newpage

\begin{center}
\bfseries НАЦИОНАЛЬНЫЙ ИССЛЕДОВАТЕЛЬСКИЙ ЯДЕРНЫЙ УНИВЕРСИТЕТ \guillemotleft МИФИ\guillemotright \\
\vspace{0.7cm}
\bfseries ФАКУЛЬТЕТ КИБЕРНЕТИКИ И ИНФОРМАЦИОННОЙ БЕЗОПАСНОСТИ \\*
КАФЕДРА \guillemotleft КОМПЬЮТЕРНЫЕ СИСТЕМЫ И ТЕХНОЛОГИИ\guillemotright \\*
\end{center}
 
Специальность 230101 \hfill Группа В7-123

\vspace{2em}

\hfill\begin{minipage}[t]{6cm}
\begin{center} 
	\guillemotleft \uppercase{Утверждаю}\guillemotright \\
	Заведующий кафедрой
\end{center}
\begin{flushleft}
\rule{3cm}{0.4pt} М.А. Иванов
"\rule{.5cm}{0.4pt}" \rule{3cm}{0.4pt} 2013 г.
\end{flushleft}
\end{minipage}


\vspace{8em}

\begin{center}
\large\bfseries ЗАДАНИЕ НА ВЫПУСКНУЮ КВАЛИФИКАЦИОННУЮ РАБОТУ \\ (ДИПЛОМНЫЙ ПРОЕКТ)
\end{center}

\vspace{3em}
 
\begin{flushleft}
Фамилия, имя, отчество студента: \textbf{\username} \\*
Тема работы: \begin{minipage}[t]{0.75\textwidth} {\textbf{\thesisTheme}} \end{minipage}
\end{flushleft}
Руководитель работы: \textbf{\projectManager} \\*
Место выполнения: \textbf{\workplace}

\vspace{\fill}

\begin{center}
Москва 2014
\end{center}

\end{titlepage}% это титульный лист
\tableofcontents % это оглавление, которое генерируется автоматически
\newpage
\chapter*{\centerline{Аннотация}}

В аннотации проекта (объем до 1 страницы) кратко излагается содержание разделов
пояснительной записки (например, "Раздел 2 посвящен выбору элементной базы проектируемой
системы. С учетом требований к производительности системы выбран микроконтроллер типа ...").
% аннотация

\chapter*{Техническое задание}
	\addcontentsline{toc}{chapter}{Техническое задание}
	\begin{enumerate}
	\item Исходные данные:
	\mbox{}\\Разрабатываемая система предназначена для стабилизации восьмимоторного квадрокоптера на базе микроконтроллера ATMEGA328P-PU с использованием библиотек Arduino. Моторы квадрокоптера предусматривают вращение только в одну сторону. Система должна стабилизировать полет квадрокоптера.
	\item Содержание задания:
	\begin{enumerate}
		\item\itshape литература и обзор работ, связанных с темой работы
		\item\itshape расчетно-конструкторская, теоретическая, технологическая части
		\item\itshape экспериментальная часть
	\end{enumerate}
	\item Основная литература
	\item Отчетный материал:
		\begin{enumerate}
		\item{\itshape пояснительная записка}
		\item{\itshape макетно-экспериментальная часть:}
			\begin{enumerate}
			\item Листинги отлаженных программ
			\item Материалы отладки
			\item Дистрибутив системы на CD
			\item Инструкция пользователя
			\end{enumerate}
		\end{enumerate}
\end{enumerate}

\vfill %отступаем до конца страницы
%\begin{center}
%	\begin{minipage}[b]{12cm}
%		\begin{flushleft}
%		Дата выдачи задания: 15 октября 2013 г. \\
%		Задание принял к исполнению \rule{3cm}{0.4pt} \\
%		Руководитель \rule{6cm}{0.4pt} \\
%		Консультант \rule{6cm}{0.4pt} \\
%		\end{flushleft}
%	\end{minipage}
%\end{center}% текст технического задания

\chapter*{Введение}
	\addcontentsline{toc}{chapter}{Введение}

\chapter{Обзорная часть}
	\section{Обзор существующих моделей}
	\section{Супер таблицы производительности}
	\section{test}

\chapter{Реализация}
	\section{Код}
	\section{Объяснение}
	
\chapter{Тестирование}
	\section{JUnit}
	\section{Тестирование реальной модели}

\chapter*{Заключение}
	\addcontentsline{toc}{chapter}{Заключение}

\chapter*{Литература}
	\addcontentsline{toc}{chapter}{Литература}

\chapter*{Приложение}
	\addcontentsline{toc}{chapter}{Приложение}
	\section{Программа отладки на Java}
	\section{Программа управления на C}

\end{document}